\chapter{Návrh metody detekce fyzické aktivity}

Při fyzické zátěži dochází k poklesu cukru v krvi. Její detekcí a snížením množství podávaného bazálu inzulinu lze snížit riziko hypoglykémie.

Senzor CGMS měří buď srdeční tep, počet kroků a elektrodermální aktivitu v pětiminutových intervalech nebo akceleraci a elektrodermání aktivitu v minutových intervalech (viz kapitola \ref{ch:bglp}). Tyto hodnoty se zvyšují zvláště u aerobních cvičení. Při posilovacích cvičeních nemusí být tak markantní.

Srdeční tep se v klidovém stavu pohybuje mezi 60 - 80 tepy za minutu, při zátěži pak mezi 100 - 180 tepy za minutu dle typu zátěže. U trénovaných sportovců jsou tyto hodnoty nižší. Srdeční tep zvyšují i další vlivy jako je stres nebo vypití kofeinového nápoje.

Fyzická aktivita je často určitá forma pohybu. Ten je měřen buď akcelerací nebo počtem kroků. U aerobních cvičení budou změny v pohybu značné. U posilování, kdy je člověk většinou na místě, budou zanedbatelné. Vyšší hodnoty budou také při běžné chůzi.

Elektrodermální aktivita je kožně galvanická reakce, tj. udává vodivost kůže. Při cvičení dochází k pocení, čímž se elektrodermální aktivita pokožky zvyšuje. K tomu dochází až se zpožděním. Zvýšení může být opět v důsledku více vlivů.

Navržený algoritmus detekuje fyzickou aktivitu na základě těchto ukazatelů. Pokud hodnota překročí daný threshold, je detekována fyzická aktivita. Použít se může buď aktuální hodnota nebo průměr za určité časové okno. Thresholdy jsou individuální pro každého pacienta (především srdeční tep se u každého liší).  Pro snížení falešně pozitivních detekcí je vhodné vytvořit kombinace ukazatelů, které se budou podílet na detekci. Aby byla fyzická aktivita detekována musí být všechny nad svým thresholdem. Tím se eliminují vlivy, které mají vliv pouze na jeden ukazatel.

Jelikož při fyzické aktivitě dochází k poklesu hladiny glukózy, může se detekce zpřesnit detekcí sestupné hrany. K tomu lze použít algoritmus \textit{Detekce hran průběhu intersticiální glukózy} navržený v kapitole \ref{ch:threshold}. Pro detekci sestupné hrany jsou thresholdy a váhy záporné. Detekci sestupné hrany není možné použít samostatně, protože k poklesu krevního cukru dochází i v důsledku podání vyššího množství inzulinu.\enlargethispage{\baselineskip} %keep last line on the page