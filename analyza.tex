\chapter{Analýza metod detekce příjmu karbohydrátů}

Metody detekce karbohydrátů mohou být model-based nebo data-driven.

Většina studií je založená na implementaci fyzického modelu (většinou Bergmanův minimální model) a aplikaci predikčního algoritmu jako je například Kalmanův filtr pro predikci jednotlivých stavů (glykémie a karbohydráty). Detekce karbohydrátů je pak porovnáním pozorovaných stavů a modelu vůči definovanému thresholdu, výpočtu cross-covariance nebo aplikováním rozhodovacích pravidel.

U data-driven metod je extrakce vlastností kvantitativní nebo kvalitativní. Kvantitativní metoda je například analýza hlavních komponent. V kvalitativních modelech jsou časová data převedena na sekvenci kvalitativních proměnných k vytvoření kvalitativní reprezentace dat. Data-driven metody jsou méně závislé na přesnosti fyzického modelu, ale je potřeba pro jejich natrénování velkého množství vzorků. Mezi data-driven metody patří i neuronové sítě.

\section{Bergmanův minimální model}

\section{Metody založené na porovnání měřených hodnot vůči modelu}