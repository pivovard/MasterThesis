\chapter*{Příloha B}
\addcontentsline{toc}{chapter}{Příloha B}

\section*{Adresářová struktura}
\addcontentsline{toc}{section}{Adresářová struktura}

\begin{itemize}
\item Aplikace\_a\_knihovny
 \begin{itemize}
  \item detection
  \begin{itemize}
  \item src - zdrojové soubory filtrů pro SmartCGMS
  \item CMakeLists.txt
  \end{itemize}
 \item lib - knihovny
 \item scripts - python skripty
 \item setup - konfigurační soubory
 \item detection.dll - knihovna filtrů pro SmartCGMS
 \end{itemize}
\item Poster
\item Text\_prace
 \begin{itemize}
  \item src - zdrojové tex soubory
  \item dp.pdf
  \end{itemize}
\item Vstupni\_data

 \begin{itemize}
  \item model - natrénované keras modely GRU sítě\\(vstupní data pacientů není možné přiložit z důvodu autorských práv)
  \end{itemize}
\item Vysledky - vysledky provedených experimentů
\end{itemize}

\section*{Knihovny, kompilace, spuštění}
\addcontentsline{toc}{section}{Knihovny, kompilace, spuštění}

SmartCGMS je dostupný na https://diabetes.zcu.cz/smartcgms. Knihovna je
kompilována v C++ 17. Pro kompilaci jsou nutné header soubory SmartCGMS
a knihovny \href{https://kinddragon.github.io/vld/}{VisualLeakDetector},
\href{https://github.com/Dobiasd/frugally-deep}{frugally-deep},
\href{https://gitlab.com/libeigen/eigen}{Eigen},
\href{https://github.com/Dobiasd/FunctionalPlus}{FunctionalPlus} a
\href{https://github.com/nlohmann/json}{json}. K dispozici je CMake
skript. Vytvořená dynamická knihovna \texttt{detection.dll} se umístí do
složky filters. Grafické rozhraní SmartCGMS se spustí programem
\texttt{gpredict3.exe}, konzolová verze programem \texttt{console3.exe}.

\section*{Nastavení filtrů}
\addcontentsline{toc}{section}{Nastavení filtrů}

\subsection*{Savitzky-Golay filtr}

Filtr pro vyhlazení dat.

\begin{itemize}
\setlength\itemsep{-1.5em}
\item Signal - zdrojový signál pro vyhlazení\\
\item Window size - velikost okna Savitzky-Golay filtru\\
\item Degre - stupeň polynomu
\end{itemize}

\noindent Filtr posílá vyhlazená data v signálu Savgol signal. V případě použití více filtrů je nutné výstupní signál přemapovat.

\subsection*{CHO detection}

Filtr detekce příjmu karbohydrátů.

\begin{itemize}
\setlength\itemsep{-1.5em}
\item Signal - detekovaný signál\\
\item Window size - velikost klouzavého okénka\\
\item Detect edges - detekce vzestupných hran\\
\item Detect descending edges - detekce sestupných hran\\
\item Rise threshold - threshold pro určení míry stoupání/klesání v čase\\
\item Use RNN - použití rekurentní neuronové sítě\\
\item RNN model file path - cesta k souboru s natrénovaným keras modelem převedeným do formátu pro frugally-deep\\
\item RNN threshold - threshold detekce neuronovou sítí\\
\item Thresholds\\
	\indent- Threshold Low - threshold malé změny IST\\
	\indent- Weight Low - váha malé změny IST\\
	\indent- Threshold High - threshold velké změny IST\\
	\indent- Weight High - váha velké změny IST
\end{itemize}

\noindent Filtr posílá aktivační funkce a detekované karohydráty.
Příklad konfigurace detekce hran průběhu intersticiální glukózy je v souboru \texttt{setup/setup\_th.ini}, příklad neuronové sítě v souboru \texttt{setup/setup\_gru.ini`}.

\subsection*{PA detection}

Filtr detekce fyzické aktivity.

\begin{itemize}
\setlength\itemsep{-1.5em}
\item Heartbeat - detekce podle srdečního tepu\\
\item Steps - detekce podle počtu kroků\\
\item Acceleration - detekce podle hodnoty akcelerace\\
\item Electrodermal activity - detekce podle elektrodermální aktivity\\
v Mean - použití průměru za časové okno\\
\item Window size - mean - velikost klouzavého okénka pro spočítání průměru (v případě velikosti okna 1 je průměr rovná aktuální hodnotě)\\
\item Detect IST edges - potvrzení detekce sestupnou hranou dat IST\\
\item Signal - detekovaný signal\\
\item Window size - edges - velikost klouzavého okénka pro detekci hran\\
\item Thresholds - threshold ukazatelů pro detekci a thresholdy a váhy pro detekci hran
\end{itemize}

\noindent Filtr posílá detekovanou fyzickou aktivitu.
Příklad konfigurace s měřeným srdečním tepem a počtem kroků je v konfiguračním souboru \texttt{setup/setup\_bpm.ini}. Příklad konfigurace s akcelerací a potvrzováním pomocí detekce sestupné hrany je v konfiguračním souboru \texttt{setup/setup\_acc.ini}.

\subsection*{Evaluation}

Filtr pro vyhodnocení výsledků.

\begin{itemize}
\setlength\itemsep{-1.5em}
\item Reference signal - referenční signál\\
\item Detected signál - detekovaný signál\\
\item Max detection delay - maximální zpoždění, které může mít detekovaný signál oproti referenčnímu\\
\item False positive cooldown - cooldown po detekci falešně pozitivního signálu, než je započítán další\\
\item Late detection delay - čas před referenčním signálem, kdy se bude detekce počítat jako pravdivě pozitivní\\
\item Min reference count - minimální počet referenčních signálů za den
\end{itemize}

\noindent Filtr na konci běhu simulace posílá info s naměřenými statistikami počtu referenčních signálů TP, potvrzené TP, FN, FP, zpoždění detekce a zpoždění potvrzení.

\section*{Python skripty}
\addcontentsline{toc}{section}{Python skripty}

Skripty umožňují transformaci a modifikaci dat BGLP, trénování
rekurentní neuronové sítě a analýzu metod detekce karbohydrátů a fyzické
aktivity. Pro spuštění skriptů je nutné mít nainstalovaný Python 3.7
nebo 3.8. Python 3.9 není podporovaný z důvodu nepodporované
kompatibilitě s knihovnou TensorFlow. Dále je nutné mít nainstalované
tyto balíčky: numpy, pandas, scipy, sklearn, tensorflow, tabulate, matplotlib, sweetviz.

V souboru \texttt{examples.py} jsou příklady práce se skripty:

\begin{itemize}
\itemsep1pt\parskip0pt\parsep0pt
\item lg.load\_log()- transformace dat z logu do csv
\item ld.load\_data() - modifikace dat pro detekci karbohydrátů nebo fyzické aktivity
\item cho.rnn() - trénování RNN
\item cho.lda() - LDA/GDA
\item cho.threshold() - detekce hran IST
\item pa.get\_feature() - spočtení vlastností (průměr, medián, směrodatná odchylka, rozdíl kvartilů)
\item pa.export\_features() - export vlastností do csv
\item pa.ML() - test metod strojového učení pro detekci fyzické aktivity
\end{itemize}

\subsection*{Trénování neuronové sítě}

Pro natrénování keras modelu rekurentní neuronové sítě je připraven skript \texttt{train\_rnn.py}. Ten se spouští příkazem \texttt{python train\_rnn.py\textless{}type\textgreater{} \textless{}options\textgreater{} {[}values{]}}, kdy \texttt{type} je typ neuronové \texttt{-gru} nebo \texttt{-lstm} (povinný parametr) a \texttt{options} jsou volitelné parametry: 

\noindent -f ~ název logu pacienta (* pro natrénování modelu na všech logách v adresáři)\\
 -i ~ adresář se vstupními logy (defaultně data/)\\
 -o ~ adresář pro uložení natrénovaného modelu (defaultně model/)\\
 -h ~ nápověda

\noindent Příklad natrénování GRU pro všechny pacienty: \texttt{python train\_rnn.py -gru -f * -i data/ -o model/} Příklad natrénování LSTM pro pacienta 575: \texttt{python train\_rnn.py -lstm -f 575-ws-training.log -i data/ -o model/}