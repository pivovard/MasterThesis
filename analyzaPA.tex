\chapter{Analýza metod detekce fyzické aktivity}

Fyzická aktivita výrazně snižuje koncentraci glukózy v krvi. Z toho důvodu je před zahájením cvičení snížit dávkování inzulinu, jinak hrozí riziko hypoglykémie. Systémy kontinuální monitorace glukózy jsou sice schopny signalizovat pokles hladiny glukózy, ale tato detekce nemusí nastat včas, nebo není dostatečná. Proto je třeba detekovat parametry fyzické aktivity jako takové.

Většina publikovaných metod pro detekci využívá data ze senzorů pro měření srdeční aktivity, teploty kůže, vodivosti kůže a akcelerometrů pro snímání pozice a pohybu těla. Tyto senzory mohou být externí, nebo součástí CGMS.


\section{•}

Autoři této metody využívají zařízení SenseWear® Pro Armband vyvinuté firmou BodyMedia (Pittsburgh, PA, USA). Toto zařízení je upevněno kolem ruky a sbírá data z pěti typů senzorů. Senzory snímají pozici a pohyb ruky a těla, teplotu kůže a okolí, vodivost kůže a srdeční aktivitu.