\chapter*{Příloha B}
\addcontentsline{toc}{chapter}{Příloha B}

\section*{Knihovny, kompilace, spuštění}
\addcontentsline{toc}{section}{Knihovny, kompilace, spuštění}

SmartCGMS je dostupný na \textit{https://diabetes.zcu.cz/smartcgms}. Knihovna je kompilována v C++ 17. Pro kompilaci jsou nutné header soubory SmartCGMS a knihovny VisualLeakDetector, frugally-deep, Eigen, FunctionalPlus a json. K dispozici je CMake skript.
Vytvořená dynamická knihovna \texttt{detection.dll} se umístí do složky filters. Grafické rozhraní SmartCGMS se spustí programem \texttt{gpredict3.exe}, konzolová verze programem \texttt{console3.exe}.

\section*{Nastavení filtrů}
\addcontentsline{toc}{section}{Nastavení filtrů}

\subsection*{Savitzky-Golay filtr}

Filtr pro vyhlazení dat.

\noindent- Signal - zdrojový signál pro vyhlazení\\
- Window size - velikost okna Savitzky-Golay filtru\\
- Degre - stupeň polynomu

Filtr posílá vyhlazená data v signálu Savgol signal. V případě použití více filtrů je nutné výstupní signál přemapovat.

\subsection*{CHO detection}

Filtr detekce příjmu karbohydrátů.

\noindent- Signal - detekovaný signál\\
- Window size - velikost klouzavého okénka\\
- Detect edges - detekce vzestupných hran\\
- Detect descending edges - detekce sestupných hran\\
- Rise threshold - threshold pro určení míry stoupání/klesání v čase\\
- Use RNN - použití rekurentní neuronové sítě\\
- RNN model file path - cesta k souboru s natrénovaným keras modelem převedeným do formátu pro frugally-deep\\
- RNN threshold - threshold detekce neuronovou sítí\\
- Thresholds\\
	\indent- Threshold Low - threshold malé změny IST\\
	\indent- Weight Low - váha malé změny IST\\
	\indent- Threshold High - threshold velké změny IST\\
	\indent- Weight High - váha velké změny IST

Filtr posílá aktivační funkce a detekované karohydráty.
Příklad konfigurace detekce hran průběhu intersticiální glukózy je v souboru \texttt{setup/setup\_th.ini}, příklad neuronové sítě v souboru \texttt{setup/setup\_gru.ini`}.

\subsection*{PA detection}

Filtr detekce fyzické aktivity.

\noindent- Heartbeat - detekce podle srdečního tepu\\
- Steps - detekce podle počtu kroků\\
- Acceleration - detekce podle hodnoty akcelerace\\
- Electrodermal activity - detekce podle elektrodermální aktivity\\
- Mean - použití průměru za časové okno\\
- Window size - mean - velikost klouzavého okénka pro spočítání průměru (v případě velikosti okna 1 je průměr rovná aktuální hodnotě)\\
- Detect IST edges - potvrzení detekce sestupnou hranou dat IST\\
- Signal - detekovaný signal\\
- Window size - edges - velikost klouzavého okénka pro detekci hran\\
- Thresholds - threshold ukazatelů pro detekci a thresholdy a váhy pro detekci hran

Filtr posílá detekovanou fyzickou aktivitu.
Příklad konfigurace s měřeným srdečním tepem a počtem kroků je v konfiguračním souboru \texttt{setup/setup\_bpm.ini}. Příklad konfigurace s akcelerací a potvrzováním pomocí detekce sestupné hrany je v konfiguračním souboru \texttt{setup/setup\_acc.ini}.

\subsection*{Evaluation}

Filtr pro vyhodnocení výsledků.

\noindent- Reference signal - referenční signál\\
- Detected signál - detekovaný signál\\
- Max detection delay - maximální zpoždění, které může mít detekovaný signál oproti referenčnímu\\
- False positive cooldown - cooldown po detekci falešně pozitivního signálu, než je započítán další\\
- Late detection delay - čas před referenčním signálem, kdy se bude detekce počítat jako pravdivě pozitivní\\
- Min reference count - minimální počet referenčních signálů za den

Filtr na konci běhu simulace posílá info s naměřenými statistikami počtu referenčních signálů TP, potvrzené TP, FN, FP, zpoždění detekce a zpoždění potvrzení.


\section*{Python skripty}
\addcontentsline{toc}{section}{Python skripty}

\begin{setlength}{\parskip}{0pt}
Skripty umožňují transformaci a modifikaci dat BGLP, trénování rekurentní neuronové sítě a analýzu metod detekce karbohydrátů a fyzické aktivity. V souboru `main.py` jsou příklady spuštění jednotlivých skriptů.
Pro spuštění skriptů je nutné mít nainstalovaný Python 3.7 nebo 3.8. Python 3.9 není podporovaný z důvodu nepodporované kompatibilitě s knihovnou TensorFlow. Dále je nutné mít nainstalované tyto balíčky:

\begin{itemize}
\setlength\itemsep{0em}
\item numpy
\item pandas
\item scipy
\item sklearn
\item tensorflow
\item tabulate
\item matplotlib
\item sweetviz
\end{itemize}
\end{setlength}

\subsection*{Trénování neuronové sítě}

Pro natrénování keras modelu rekurentní neuronové sítě je připraven skript \texttt{train\_rnn.py}.  Ten se spouští příkazem \texttt{python train\_rnn.py <type> <option> [IDs]}, kdy \texttt{type} je typ neuronové \texttt{-gru} nebo \texttt{-lstm}, \texttt{option} určuje zda budeme trénovat pro každého pacienta individuální model \texttt{-i} nebo společný model \texttt{-a}. \texttt{IDs} je pole id pacientů. Vstupní soubory musí být v adresáři \texttt{/model} a být pojmenované \texttt{[ID]-ws-training.log}.