\chapter{Úvod}

Diabetes mellitus je rozšířené chronické metabolické onemocnění. Vyznačuje se zvýšenou koncentrací cukru v krvi (glykémie), která vniká z nedostatku inzulinu nebo rezistencí vůči jeho působení. Hlavní součástí léčby je monitorace koncentrace glukózy a podávání inzulinu. V dnešní době je možné sledovat vývoj glykémie díky sytémům kontinuální monitorace glukózy (CGMS) a kontinuálnímu dávkování inzulinu díky inzulinovým pumpám. Integrace těchto dvou systémů umožňuje autonomní řízení dávkování inzulinu v závislosti na aktuální koncentraci glukózy.

Koncentraci glukózy v krvi ovlivňuje mnoho faktorů. Dva nejvýznamnější faktory jsou příjem karbohydrátů a fyzická aktivita. Tyto aktivity je třeba kompenzovat zvýšením či snížením množství podávaného inzulinu. Aktuálně je do CGMS zadává pacient a dávku inzulinu upravuje také on. Monitorace těchto aktivit slouží pro správné nastavení léčby diabetologem. Jejich včasná detekce by umožnila přesnější monitoraci vývoje glykémie a vznik plně autonomních systémů s minimálním zásahem pacienta.

V mé práci se budu zabývat možnostmi detekce příjmu kabohydrátů a fyzické aktivity z dat senzoru CGMS. Následně tyto metody implementuji do aplikace SmartCGMS vyvíjené na katedře informatiky a výpočetní techniky na ZČU.